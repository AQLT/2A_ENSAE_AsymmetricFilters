% Options for packages loaded elsewhere
\PassOptionsToPackage{unicode}{hyperref}
\PassOptionsToPackage{hyphens}{url}
%
\documentclass[
  12pt,
  french]{article}
\usepackage{lmodern}
\usepackage{amssymb,amsmath}
\usepackage{ifxetex,ifluatex}
\ifnum 0\ifxetex 1\fi\ifluatex 1\fi=0 % if pdftex
  \usepackage[T1]{fontenc}
  \usepackage[utf8]{inputenc}
  \usepackage{textcomp} % provide euro and other symbols
\else % if luatex or xetex
  \usepackage{unicode-math}
  \defaultfontfeatures{Scale=MatchLowercase}
  \defaultfontfeatures[\rmfamily]{Ligatures=TeX,Scale=1}
\fi
% Use upquote if available, for straight quotes in verbatim environments
\IfFileExists{upquote.sty}{\usepackage{upquote}}{}
\IfFileExists{microtype.sty}{% use microtype if available
  \usepackage[]{microtype}
  \UseMicrotypeSet[protrusion]{basicmath} % disable protrusion for tt fonts
}{}
\makeatletter
\@ifundefined{KOMAClassName}{% if non-KOMA class
  \IfFileExists{parskip.sty}{%
    \usepackage{parskip}
  }{% else
    \setlength{\parindent}{0pt}
    \setlength{\parskip}{6pt plus 2pt minus 1pt}}
}{% if KOMA class
  \KOMAoptions{parskip=half}}
\makeatother
\usepackage{xcolor}
\IfFileExists{xurl.sty}{\usepackage{xurl}}{} % add URL line breaks if available
\IfFileExists{bookmark.sty}{\usepackage{bookmark}}{\usepackage{hyperref}}
\hypersetup{
  pdftitle={Rapport},
  pdfauthor={Alain Quartier-la-Tente},
  hidelinks,
  pdfcreator={LaTeX via pandoc}}
\urlstyle{same} % disable monospaced font for URLs
\usepackage[margin=1in]{geometry}
\usepackage{longtable,booktabs}
% Correct order of tables after \paragraph or \subparagraph
\usepackage{etoolbox}
\makeatletter
\patchcmd\longtable{\par}{\if@noskipsec\mbox{}\fi\par}{}{}
\makeatother
% Allow footnotes in longtable head/foot
\IfFileExists{footnotehyper.sty}{\usepackage{footnotehyper}}{\usepackage{footnote}}
\makesavenoteenv{longtable}
\usepackage{graphicx,grffile}
\makeatletter
\def\maxwidth{\ifdim\Gin@nat@width>\linewidth\linewidth\else\Gin@nat@width\fi}
\def\maxheight{\ifdim\Gin@nat@height>\textheight\textheight\else\Gin@nat@height\fi}
\makeatother
% Scale images if necessary, so that they will not overflow the page
% margins by default, and it is still possible to overwrite the defaults
% using explicit options in \includegraphics[width, height, ...]{}
\setkeys{Gin}{width=\maxwidth,height=\maxheight,keepaspectratio}
% Set default figure placement to htbp
\makeatletter
\def\fps@figure{htbp}
\makeatother
\setlength{\emergencystretch}{3em} % prevent overfull lines
\providecommand{\tightlist}{%
  \setlength{\itemsep}{0pt}\setlength{\parskip}{0pt}}
\setcounter{secnumdepth}{5}
\usepackage{setspace}
\onehalfspacing
\usepackage{mathptmx}
\usepackage{multicol}

\title{Rapport}
\author{Alain Quartier-la-Tente}
\date{6/17/2020}

\begin{document}
\maketitle

{
\setcounter{tocdepth}{2}
\tableofcontents
}
\hypertarget{les-diffuxe9rents-noyaux}{%
\subsection{Les différents noyaux}\label{les-diffuxe9rents-noyaux}}

Dans l'extraction du signal, on cherche généralement à pondérer les observations selon leur distance à la date \(t\).
Pour cela on introduit une fonction de noyau \(\kappa_j\), \(j=0,\pm1,\dots,\pm h\) avec \(\kappa_j \geq0\) et \(\kappa_j=\kappa_{-j}\).
Une classe importante des noyaux qui comprend la majorité de ceux utilisé est la classe des Beta kernels :
\[
\kappa(u)=k_rs\left(1-\lvert u\rvert^r
\right)^s
1_{\lvert u\rvert\leq 1}
,\quad k_{rs}=\frac{r}{
2B\left(s+1,\frac 1 r\right)
}
\]
Avec \(r>0\), \(s\geq 0\) et
\[
B(a,b)=\int_0^1u^{a-1}(1-u)^{b-1}du
\]
Dans le cas discret, à un facteur proportionnel près (constante de normalisation pour que \(\sum\kappa_j=1\)) :
\[
\kappa_j = \left(
  1-
  \left\lvert
  \frac j {h+1}
  \right\lvert^r
\right)^s
\]

\begin{multicols}{2}
\begin{itemize}
\item $r=1,s=0$ uniform kernel : 
$$\kappa_j^U=1$$
%$$\kappa_j^U=\frac{1}{2h+1}$$

\item $r=s=1$ triangle kernel :
$$\kappa_j^T=\left(
  1-
  \left\lvert
  \frac j {h+1}
  \right\lvert
\right)$$
%$$\kappa_j^T=\frac{\left(
%  1-
%  \left\lvert
%  \frac j {h+1}
%  \right\lvert
%\right)}{
%h+1
%}
%$$

\item $r=2,s=1$  Epanechnikov (ou Parabolic) kernel :
$$\kappa_j^E=\left(
  1-
  \left\lvert
  \frac j {h+1}
  \right\lvert^2
\right)$$
\end{itemize}
\end{multicols}

\begin{multicols}{2}
\begin{itemize}

\item $r=s=2$ biweight kernel :
$$\kappa_j^{BW}=\left(
  1-
  \left\lvert
  \frac j {h+1}
  \right\lvert^2
\right)^2$$

\item $r = 2, s = 3$ triweight kernel :
$$\kappa_j^{TW}=\left(
  1-
  \left\lvert
  \frac j {h+1}
  \right\lvert^2
\right)^3$$

\item $r = s = 3$ tricube kernel :
$$\kappa_j^{TR}=\left(
  1-
  \left\lvert
  \frac j {h+1}
  \right\lvert^3
\right)^3$$

\item Gaussian kernel  :
$$
\kappa_j^G=\exp\left(
-\frac{
  j^2
}{
  4h^2
}\right)
$$
\item Henderson kernel :
$$
\kappa_{j}=\left[(h+1)^{2}-j^{2}\right]\left[(h+2)^{2}-j^{2}\right]\left[(h+3)^{2}-j^{2}\right]
$$

\item Trapezoidal kernel :
$$
\kappa_j=
\begin{cases}
  \frac{1}{3(2h-1)} & \text{ si }j=\pm h 
  \\
  \frac{2}{3(2h-1)} & \text{ si }j=\pm (h-1)\\
  \frac{1}{2h-1}& \text{ sinon}
\end{cases}
$$

\end{itemize}
\end{multicols}

\end{document}
