%début contraintes ensae
\usepackage{pdfpages, setspace, mathptmx} %times roman
\usepackage{textpos}%pour textblock
\onehalfspacing 
\definecolor{rougeENSAE}{RGB}{188, 24, 39}

\makeatletter
%CREDITS : @antuki
\def\@maketitle{%
  \clearpage
 \thispagestyle{empty}

\begin{textblock*}{\textwidth}(-7cm,-3.5cm)
\begin{center}
\includegraphics[height=3cm]{img/900px-LOGO-ENSAE.png}
\end{center}
\end{textblock*}

\begin{minipage}{0.4\textwidth}
  \begin{flushleft} \large
    \textbf{Alain \textsc{Quartier-la-Tente} \vspace{8.75mm} }
  \end{flushleft}
\end{minipage}
\begin{minipage}{0.6\textwidth}
  \begin{flushright} 
  \large{
    \textbf{ENSAE 2\textsuperscript{ème} année\\}
  }     
  \small
    \textbf{ 
        \textit{Stage d'application\\
        Année scolaire 2019 - 2020}
    }
  \end{flushright}
\end{minipage}

\vspace*{5cm}

\begin{center}
    \fbox{\parbox{0.9\textwidth}{
        \begin{huge}\begin{center}
        \textbf{Real-time detection of turning points with linear filters}\\
        
        %\textbf{Détection en temps réel des points de retournement}\\
        \end{center}\end{huge}}}
\end{center}

\vfill
    
\begin{minipage}{0.5\textwidth}
    \begin{flushleft} \large 
    \textbf{Banque Nationale de Belgique\\
        Cellule Recherche et Développement
    }
  \end{flushleft}
\end{minipage} 
\begin{minipage}{0.5\textwidth}
  \begin{flushright} \large
    \textbf{
        Maître de stage : Jean \textsc{Palate}\\
        08/06/2020 - 14/08/2020
    }
  \end{flushright}
\end{minipage}    

\vspace*{1cm}


\textcolor{rougeENSAE}{\rule{10mm}{1.5mm}}

\scriptsize
\textbf{ENSAE Paris}\newline TSA 26644
\rightline{\href{www.ensae.fr}{\textcolor{rougeENSAE}{\textbf{www.ensae.fr}}}$\quad \qquad \qquad$}

Service des relations entreprises et des stages\newline
5, avenue Henry Le Chatelier -- 91764 PALAISEAU CEDEX -- FRANCE -- Tél : +33 (0)1 70 26 67 39 -- Courriel : stage\symbol{64}ensae.fr

\normalsize

\clearpage
\setcounter{page}{0}
}

\makeatother% cinsérer page de garde

\usepackage{stmaryrd}
\usepackage{multicol}
\usepackage{graphicx}
\usepackage{animate, dsfont, here, xspace}
%\usepackage{tikz}       
\usepackage{tikz,pgfplots}
 \pgfplotsset{compat=1.17}
%\includepdf[fitpaper=true, pages=-]{img/pdg.pdf}


\DeclareMathOperator{\e}{e}
\renewcommand{\P}{\mathds{P}} %Apparement \P existe déjà ?
\newcommand\N{\mathds{N}}
\newcommand\R{\mathds{R}}
%\newcommand\C{\mathds{C}}
%\newcommand\Z{\mathds{Z}}


\newcommand\1{\mathds{1}}
\newcommand{\E}[2][]{{\mathds{E}}_{#1}
  \def\temp{#2}\ifx\temp\empty
  \else
    \left[#2\right]
  \fi
}
\newcommand{\V}[2][]{{\mathds{V}}_{#1}
  \def\temp{#2}\ifx\temp\empty
  \else
    \left[#2\right]
  \fi
}
\newcommand\ud{\,\mathrm{d}}

% blocks
\usepackage{environ}
\usepackage[tikz]{bclogo}

\tikzstyle{titlestyle} =[draw=black!80,fill=black!20, text=black,
 right=10pt, rounded corners]
\mdfdefinestyle{symmaryboxstyle}{
	linecolor=black!80, backgroundcolor = black!5,
	skipabove=\baselineskip, innertopmargin=\baselineskip,
	innerbottommargin=\baselineskip,
	userdefinedwidth=\textwidth,
	middlelinewidth=1.2pt, roundcorner=5pt,
	skipabove={\dimexpr0.5\baselineskip+\topskip\relax},
	frametitleaboveskip=\dimexpr-\ht\strutbox\relax,
	innerlinewidth=0pt,
}
\NewEnviron{summary}[1]{%
\begin{mdframed}[style=symmaryboxstyle,
frametitle={%
      \tikz[baseline=(current bounding box.east),outer sep=0pt]
      \node[titlestyle,anchor=east]
    {Summary --- #1};}
]
\vspace{-0.5em}
\BODY
\end{mdframed}
}

\usepackage{amsthm}
%\theoremstyle{remark}
\newtheorem*{remark}{Remark}

\usepackage{mathrsfs}

